\documentclass[12pt]{article}
\usepackage[margin=3cm]{geometry}
\usepackage{feynr}
\usepackage{listings}
\usepackage{caption,subcaption}
\newcommand{\feynR}{\texttt{feynr}}
\title{Using \feynR}
\author{Peter Schmidt-Nielsen}
\begin{document}
\maketitle
\section*{Basic Usage}
\subsection*{Command-line Usage}
Include \feynR{} with the usual \texttt{\textbackslash{}usepackage\{feynr\}}.
Place your diagram source code in a \feynR{} environment to render a diagram.
For example, \texttt{\textbackslash{}begin\{feynr\} source \textbackslash{}end\{feynr\}}.
Run \texttt{feynr\_render} on your \LaTeX{} file after each time you update a \feynR{} diagram:
\begin{quote}
\texttt{feynr\_render source.tex}\\
\texttt{pdflatex source.tex}
\end{quote}
You can download \feynR{} at

In \feynR{}, particles are explicitly tracked.
You can create an input particle to your diagram with the \texttt{input} command.
For example, \texttt{input a} defines an input particle called \texttt{a}.
Special flags can be passed to any \feynR{} command with a pair of dashes.
For example, to declare an input particle \texttt{a} that's an electron we may use: \texttt{input a -- electron}.
More than one particle may be declared in a single command, and the flags passed apply to all particles.
For example, to declare three positrons: \texttt{input p1 p2 p3 -- positron}.
Particle types may be omitted if they can be unambiguously inferred from elsewhere in the diagram.

Particles can be made to interact with each other with the \texttt{interact} command.\footnote{The command \texttt{interact} can also be abbreviated as a single dash.}
For example, \texttt{interact p1 p2} will cause the particles \texttt{p1} and \texttt{p2} to exchange a particle.
Again, the interaction type may be unambiguous.
For example, if \texttt{p1} and \texttt{p2} are photons, and \texttt{p1} is already known to emit a positron (from another \texttt{interact} command), then the interaction is unambiguous, and an electron will be emitted from \texttt{p1}, and propagate to interact with \texttt{p2}.
However, frequently you will have to specify what form the interaction takes.
The flags \texttt{electron}, \texttt{positron}, and \texttt{photon} denote that the given particle propagated from the first argument to the second argument.
For example, \texttt{interact p1 p2 -- electron} causes an electron to be emitted by \texttt{p1} and absorbed by \texttt{p2}.\footnote{Naturally, \texttt{interact a b -- electron} is equivalent to \texttt{interact b a -- positron}.}
Note that as \texttt{interact} takes exactly two node arguments, the \texttt{--} separating flags from node arguments is optional.
You may write either \texttt{interact a b -- flags} or \texttt{interact a b flags}, these are equivalent.
Just like \texttt{input}, there is a command \texttt{output} that can be used to disambiguate diagrams.
For example, if, at the end of a sequence of interactions, it's ambiguous what type of particle \texttt{a} is, we can specify with, for example: \texttt{output a -- photon}.
You can identify interactions in your diagram that couldn't have their type inferred -- they will appear as red sine waves.

Finally, we need to give \feynR{} some hints about the layout of our diagram.
If you attempt to compile a diagram without any layout hints it will fail, as it can't figure out what angles the various lines should be drawn at.
There are two default layout hints, \texttt{0space} and \texttt{0time}.
These hints indicate that the given interaction should be drawn perpendicular to the specified axis.
In other words, in a time-up Feynman diagram, \texttt{0time} indicates that the interaction should be drawn horizontally.
For example, we can write \texttt{interact e1 e2 photon 0time} to cause particles \texttt{e1} and \texttt{e2} to exchange a photon that is drawn perpendicular to the time axis.

The \texttt{propagate} command accumulates flags to be added on the next propagation that a given particle makes.
For example, if we would like the particle \texttt{a} to propagate straight along the time axis the next time it propagates, we can issue the command \texttt{propagate a -- 0space}.

Putting these all pieces together, let's look at some examples:

\begin{figure}[h]\begin{subfigure}[h]{0.4\textwidth}\begin{center}
\begin{lstlisting}
input e -- electron
propagate e -- 0space
\end{lstlisting}
\end{center}\end{subfigure}\hfill\vrule\hfill\begin{subfigure}[h]{0.4\textwidth}\begin{center}
\begin{feynr}
input e -- electron
propagate e -- 0space
\end{feynr}
\end{center}\end{subfigure}\end{figure}
The first line defines an electron, and the second is a layout hint that it should propagate along the time axis.

\begin{figure}[h]\begin{subfigure}[h]{0.4\textwidth}\begin{center}
\begin{lstlisting}
input e1 e2 -- electron
interact e1 e2 photon 0time
\end{lstlisting}
\end{center}\end{subfigure}\hfill\vrule\hfill\begin{subfigure}[h]{0.4\textwidth}\begin{center}
\begin{feynr}
input e1 e2 -- electron
interact e1 e2 photon 0time
\end{feynr}
\end{center}\end{subfigure}\end{figure}
Here two electrons are defined in the first line, and made to exchange a photon along the space axis in the second.

\begin{figure}[h]\begin{subfigure}[h]{0.4\textwidth}\begin{center}
\begin{lstlisting}
input e -- electron
input p -- photon
interact e p 0time
\end{lstlisting}
\end{center}\end{subfigure}\hfill\vrule\hfill\begin{subfigure}[h]{0.4\textwidth}\begin{center}
\begin{feynr}
input e -- electron
input p -- photon
interact e p electron 0time
\end{feynr}
\end{center}\end{subfigure}\end{figure}
Not substantially more complicated than the previous example.

\begin{figure}[h!]\begin{subfigure}[h]{0.4\textwidth}\begin{center}
\begin{lstlisting}
input p1 p2 -- photon
interact p1 p2 electron 0time
propagate p1 p2 -- 0space
interact p1 p2 electron 0time
\end{lstlisting}
\end{center}\end{subfigure}\hfill\vrule\hfill\begin{subfigure}[h]{0.4\textwidth}\begin{center}
\begin{feynr}
input p1 p2 -- photon
interact p1 p2 electron 0time
propagate p1 p2 -- 0space
interact p1 p2 electron 0time
\end{feynr}
\end{center}\end{subfigure}\end{figure}
Here Delbr\"uck scattering is concisely described in four lines.
The third line (\texttt{propagate p1 p2 -- 0space}) makes sure that the electrons propagate vertically in the diagram, forming a nice square box when used with the horizontal \texttt{0time} interactions.

\section*{Advanced Usage}

\end{document}

